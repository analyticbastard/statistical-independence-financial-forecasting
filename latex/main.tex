	% Author : Pierre-Henri Horrein
% Copyright : Telecom Bretagne, 2013

% Définition du document
\documentclass[t,9pt,pdftex]{beamer}
\usetheme{tb}
\usepackage{ucs}
\usepackage[utf8x]{inputenc}
\usepackage{textcomp}
\usepackage{graphicx}
\usepackage{amsmath}
%\usepackage[spanish]{babel}
%\setbeamertemplate{navigation symbols}[horizontal]
\usepackage{multicol}
\usepackage{url}
\usepackage{listings}
\lstset{
basicstyle=\small\ttfamily,
columns=flexible,
breaklines=true
}

\graphicspath{{figs/}}


% Entre [], le titre raccourci pour le bas de page, 
% Entre {}, le titre complet pour la page de garde
\title[]{Independencia estad\'istica: El An\'alisis t\'ecnico de mercados financieros como caso de apofenia social}

% Nom de l'auteur
\author{Javier Arriero Pais}

% Date de la présentation (bas de page)
\date{13/07/2018}


\begin{document}

%---------- Titre -----------

% Création de la page de garde à partir des informations précédentes
\inserttitlepage


%---- Table des matières ----

% Une slide d'agenda
\begin{frame}[c]
	\frametitle{Agenda}
	% Plusieurs choix possibles (décommenter le choix sélectionné, commenter les
	% autres):
	% Uniquement les sections principales : 
	\tableofcontents[hideallsubsections]
	% Sections et sous sections : 
	%\tableofcontents
\end{frame}


\section{Definici\'on}

% On peut rappeler le plan à chaque début de section ou même de sous-section. La position de cette
% commande est importante : si on la met avant la première section, elle
% apparaîtra dans la présentation avant la première section

%\AtBeginSubsection[] {
%\AtBeginSection[] {
% \begin{frame}[c]
% \frametitle{Sommaire}
%   \small \tableofcontents[hideallsubsections,sectionstyle=show/shaded,subsectionstyle=hide/hide/hide ]
% \end{frame}
%}

\begin{frame}[c]
	\frametitle{Definici\'on}
	\begin{itemize}
		\item $P(X,Y) = P(X)P(Y)$
		\item $\varphi_{X,Y}(s,t) = \varphi_X(s)\varphi_Y(t)$
		\item $MI(X,Y)=0$
		\item $corr(f(X), g(Y))=0$, $\forall f \in C_X$, $\mbox{} g \in C_Y$
	\end{itemize}
\end{frame}


\begin{frame}[c]
	\frametitle{Definici\'on probabil\'istica}
	\begin{block}{Independencia}
		$$P(X,Y) = P(X)P(Y)$$
	\end{block}
	\begin{exampleblock}{Condicionar no afecta}
		$$P(X|Y)=P(X)$$
		$$P(Y|X)=P(Y)$$
	\end{exampleblock}
\end{frame}



\begin{frame}[c]
	\frametitle{Definici\'on arm\'onica}
	\begin{block}{Independencia}
		$$\varphi_{X,Y}(s,t) = \varphi_X(s)\varphi_Y(t)$$
	\end{block}
	
	\begin{exampleblock}{Transformada de Fourier de la funci\'on de densidad de probabilidad}
		$$\varphi_X(s) = \int_{\textbf{X}}{e^{-i s x} p(x)dx} $$
	\end{exampleblock}
\end{frame}



\begin{frame}[c]
	\frametitle{Definici\'on bajo la teor\'ia de la informaci\'on}
	\begin{block}{Independencia}
		$$MI(X,Y)=0$$
	\end{block}
	
	\begin{exampleblock}{La informaci\'on que aporta una variable sobre la otra es cero}
		$$MI(X,Y) = \int_{Y}{\int_{\textbf{X}}{ \log{ \frac{p(x,y)}{p(x) p(y)}}p(x,y) dx} dy} $$
	\end{exampleblock}
\end{frame}


\begin{frame}[c]
	\frametitle{Definici\'on funcional}
	\begin{block}{Independencia}
		$$corr(f(X), g(Y))=0, \mbox{} \forall f \in C_X, \mbox{} g \in C_Y$$
	\end{block}
	
	\begin{exampleblock}{No existe un par de transformaciones que hagan correlar las variables}
		$$corr(f(X), g(Y)) = \frac {Cov(f(X), g(Y))}{\sqrt{Var(f(X))} \sqrt{Var(g(Y))}} $$
	\end{exampleblock}
\end{frame}


\section{Predicci\'on de series temporales}

\begin{frame}[c]
	\frametitle{Predicci\'on de series temporales}
	\begin{block}{Procesos estoc\'asticos}
		$$... X_{t-2}, X_{t-1}, X_{t}, X_{t+1}, X_{t+2}, ...$$
	\end{block}
	
	\begin{exampleblock}{Modelos autorregresivos}
		$$X(t+1)=f(X(t), \epsilon)$$
	\end{exampleblock}
\end{frame}


\begin{frame}[c]
	\frametitle{Informaci\'on del futuro en el presente}	
	\begin{block}{Hip\'otesis}
		\begin{itemize}
		\item $X(t)$ contiene informaci\'on sobre $X(t+1)$
		\item $X(t)$ y $X(t+1)$ no son independientes
		\end{itemize}
	\end{block}
	
	\begin{exampleblock}{Rechazar uno de los siguientes}
		\begin{itemize}
		\item $P(X(t+1),X(t)) = P(X(t+1))P(X(t))$
		\item $\varphi_{X(t+1),X(t)}(r,s) = \varphi_{X(t+1)}(r)\varphi_{X(t)}(s)$
		\item $MI(X(t+1),X(t))=0$
		\item $corr(f(X(t+1)), g(X(t))) = 0$, $\forall f \in C_{X(t+1)}$, $g \in C_{X(t)}$
		\end{itemize}
	\end{exampleblock}
\end{frame}


\section{T\'ecnicas para medir la dependencia estad\'istica}


\begin{frame}[c,containsverbatim]
\frametitle{Clonado del repositorio}
	\begin{lstlisting}
		git clone https://github.com/analyticbastard/statistical-independence-financial-forecasting.git
	\end{lstlisting}
		
\end{frame}

\begin{frame}[c,containsverbatim]
\frametitle{Sesi\'on pr\'actica: Instalaci\'on de paquetes de CRAN y carga de datos}
	Abrir el fichero \texttt{load-data.R}
		
\end{frame}

\begin{frame}[c,containsverbatim]
\frametitle{Correlaci\'on lineal: Sesi\'on pr\'actica}
	Abrir el fichero \texttt{linear.R}
\end{frame}


\begin{frame}[c]
	\frametitle{Kernel CCA}
	$$corr(f(X), g(Y))=0$$
	\begin{block}{Generar infinitas combinaciones de funciones en $C_X$ y $C_Y$ para probar}
		\begin{itemize}
		\item Imposible
		\end{itemize}
	\end{block}	
	\begin{exampleblock}{Usar el kernel trick}
		\begin{itemize}
		\item Los RKHS son suficientemente grandes en C (densos en $L^2$)
		\item Pueden extenderse m\'etodos basados en producto interior
		\end{itemize}
	\end{exampleblock}
\end{frame}

\begin{frame}[c]
	\frametitle{Reproducing Kernel Hilbert Space}
	\begin{block}{Espacio de funciones generado de manera especial}
		\begin{itemize}
		\item $k(x,y)$ p.s.d. es el kernel del espacio $\mathbf{H}_k$
		\item Delta de Dirac (funcional de evaluaci\'on puntual) son funciones en $\mathbf{H}_k$: Se fija un par\'ametro del kernel: $k_x = k(x,\cdot)$. $f \in \mathbf{H}_k$, $f(x) = \langle f, k_x \rangle = \sum_{i=0}^{\infty}{\alpha_i k(x_i, x)}$
		\item Aproximaci\'on finita: Para un conjunto de datos $(x_1, x_2, \ldots, x_n)$, $K$ es la matriz kernel evaluando la funci\'on $k$ en los pares de puntos, $K_{ij} = k(x_i,x_j)$
		\end{itemize}
	\end{block}
\end{frame}

\begin{frame}[c]
	\frametitle{Correlaciones can\'onicas}
	\begin{block}{Buscar subespacios ortogonales de m\'axima correlaci\'on}
		\begin{itemize}
		\item Subespacio de m\'axima correlacion: $$\max_{\alpha1\in\mathbf{R}^N, \beta_1\in\mathbf{R}^M}{\mbox{corr}(X\alpha_1, Y\beta_1)}$$
		\item Siguiente par: igual, sujeto a $\alpha_1 \perp \alpha_2$ y $\beta_1 \perp \beta_2$
		\item $X$ e $Y$ pueden tener diferente dimensi\'on
		\end{itemize}
	\end{block}	
	\begin{block}{Correlaciones can\'onicas con kernel trick}
		$$
		\max_{\alpha, \beta} \frac{\alpha^T K_x K_y \beta}{\sqrt{\alpha^T K_x^2 \alpha} \sqrt{\beta^T K_y^2 \beta}}
		$$
	\end{block}
\end{frame}

\begin{frame}[c,containsverbatim]
\frametitle{KCCA: Sesi\'on pr\'actica}
	Abrir el fichero \texttt{kcca.R}
		
\end{frame}


\begin{frame}%[c,containsverbatim]
\frametitle{ForeCA}
	\begin{block}{ForeCA}
		Forecastable Components Analysis, data una serie temporal, ForeCA calcula un componente predecible y una serie residuo de ruido ortogonal a este, para lo que usa la entrop\'ia de Shannon de una m\'etrica particular (aproximaci\'on de la informaci\'on m\'utua)
	\end{block}
	
	\url{https://arxiv.org/abs/1205.4591}
		
\end{frame}


\begin{frame}[c,containsverbatim]
\frametitle{ForeCA: Sesi\'on pr\'actica}
	Abrir el fichero \texttt{foreca.R}
		
\end{frame}




\section{An\'alisis t\'ecnico}

\begin{frame}[c]
	\frametitle{Indicadores y estad\'isticos t\'ecnicos}
	
	\begin{block}{Forma general}
		$$g(X_{t+n_1}, X_{t+n_2}, \ldots, X_{t+n_N})
	 = f(X_{t-i_1}, X_{t-i_2}, \ldots, X_{t-i_L})$$		
	\end{block}
	
	\begin{exampleblock}{Pero...}
		... hemos visto que las $X_j$ son (cuasi) independientes entre ellas  
	\end{exampleblock}
	
\end{frame}



\section{Conclusiones}

\begin{frame}[c]
	\frametitle{Conclusiones}
	
		\begin{itemize}
		\item Ni el an\'alisis t\'ecnico ni una red neuronal pueden predecir futuros precios
		\item No free lunch ilustrado estad\'isticamente
		\item Para un problema de ML: Pensar antes de disparar
		\item $... X_{t-2}, X_{t-1}, X_{t}, X_{t+1}, X_{t+2}, ...$ es estacionario
		\item ML para rebalanceo de portfolio: M\'as all\'a de Markowitz
		\item Jim Simmons, Rentec, Medallion Fund
		\end{itemize}
	
\end{frame}



%\begin{frame}[c]
%	\frametitle{Deux colonnes}
%	\begin{columns}
%		\begin{column}[c]{.4\textwidth}
%			% L'environnement figure est nécessaire ici
%			\begin{figure}
%				\centering
%				\includegraphics[width=.9\textwidth]{Modele/bretagne_quadri}
%			\end{figure}
%		\end{column}
%		\begin{column}[c]{.6\textwidth}
%			\begin{itemize}
%					\vfill
%				\item on peut par exemple décrire l'image à gauche
%					\vfill
%				\item ou comparer deux colonnes
%					\vfill
%			\end{itemize}
%		\end{column}
%	\end{columns}
%\end{frame}






\end{document}
